\chapter{绪\hspace{6pt}论}

\section{研究背景及意义}
随着我国经济社会快速发展,道路建设事业也取得巨大发展,道路交通规模不断扩大,人民生活水平显著提升,
汽车保有量持续增加。交通和汽车行业快速发展的同时,车流量也显著增大,车辆超载现象严重,对路面的损害
一直存在,及时的路面病害检测和发现成为道路养护的重要工作。根据
2022年我国交通运输行业发展统计公报\citing{jtgongbao}可知,截止2022年年底,全国公路里程
535.48万公里,而公路养护里程为535.03万公里,占公路总里程比重达到99\%。

我国《公路技术状况评定标准》\citing{jtgstandard}指出,路面裂缝类病害是道路损害中常见的损害类型之一,是道路结构性问题的
先兆。未能及时检测和发现\citing{pavement_deterioration}不仅会导致其进一步恶化,降低道路使用寿命,增加后期养护成本,而且会影响乘客出行体验甚至造成严重的交通安全事故。
因此,针对道路病害尤其是路面裂缝的早期检测和识别,对于道路养护和避免交通安全事故具有重要意义。

截止目前,现有的道路裂缝检测方法主要为路面检测车和人工统计,人工检测方法不仅耗时耗力,而且具有
主观性检测误差等问题。随着计算机技术快速发展,逐渐出现了自动化裂缝检测技术,基于传统图像处理的检测方法
虽然解放了人力,但是存在检测精度不够、检测效率不佳的缺点。随着深度学习技术的出现和发展,路面裂缝检测开始
借助深度神经网络模型,不仅提高了检测效率,而且保证了一定的检测精度。

然而,现有的以深度学习为代表的裂缝检测技术,虽然解放了人力,提高了检测精度,但是仍然存在以下问题和挑战:

第一,在算法层面,目前业界提出的深度学习模型较多关注于检测精度、鲁棒性的提升,导致模型尺寸越来越大,模型参数越来越多,而忽视了模型的轻量化和检测速度的提高,这对于路面检测实际应用
场景十分不利。
路面检测模型推理的时延和帧率,直接决定了公路检测车的行驶速度,直接影响路政部门进行路面检测的工作效率。
较高的检测速度和帧率,能够提高检测车的行驶速度,提升路面裂缝检测的工作效率。

第二,在系统层面,目前业界提出的裂缝检测系统多是基于云端高算力环境进行设计和部署,公路检测车需要将采集的图像数据上传至云端进行推理和识别,
既对算力和资源要求高,提高检测的成本,又增加了检测的通信时延,还存在中心服务器负载过高和部分重要场景隐私数据泄露的隐患。因此,
亟需将推理检测模型下发至检测车本地进行部署,以降低检测成本、提高检测速度、减轻中心服务器负载和保护重要数据隐私。

针对上述两方面的难点和挑战,本文充分考虑道路裂缝检测实际场景精度和速度的双重需求,设计并实现了一套边缘智能道路裂缝检测系统。
该系统针对低成本、低功耗、资源有限的边缘设备,
在保证一定精度的前提下,设计并部署实现了基于U-Net卷积神经网络的裂缝检测算法,
以有效提升路面裂缝检测的速度和效率,且具有降低裂缝检测系统成本、减轻中心服务器负担和保护重要数据隐私的优势。

\section{国内外裂缝检测系统研究现状}
截至目前,业界提出的裂缝检测系统主要基于传统图像处理和机器学习的裂缝检测系统与基于新兴深度学习算法的裂缝检测系统。
裂缝检测系统对路面裂缝进行检测和识别的流程包括图像采集、图像预处理、裂缝检测算法处理、检测结果输出或展示等。

\subsection{基于传统图像处理和机器学习的裂缝检测系统}

Tsai等\citing{dynamic_optim}提出基于动态优化的自适应阈值分割算法,动态调整阈值实现路面损害的分割。
Wang等\citing{cracks_density_distribution}提出了基于裂缝图像密度分布的阈值分割算法,是对传统算法的优化改进。
Landstorm等\citing{morphology_method}提出了一种基于形态学方法和逻辑回归分类的裂纹检测方法,其丢弃了大多数潜在的伪缺陷(类似于缺陷的非缺陷表面特征),准确率超过80\%。
Zhao等\citing{canny_detect}提出利用Mallat小波变换来强化输入图像的弱边,对遗传算法进行二次优化,在Canny算法自适应标准中得到适当的阈值,该改进的算法可以弥补Canny算法的不足,有效地检测路面图像的边缘,并且耗时较少。
马志丹\citing{k_means_detect}提出将机器学习中的K均值聚类算法用于裂缝检测,经过比较,K-means算法优于Otsu阈值分割算法。
Waqas等\citing{svm_detect}提出以SVM为基础,为图像特征进行训练,并使用了上下文优化技术,取得了不错的效果。
Yong Shi等\citing{forest_detect}提出一种基于随机结构化森林的新型算法CrackForest,该算法有利于表示强度不均匀的裂缝,并且能从噪声中有效识别裂纹,具有良好的鲁棒性。

基于传统图像处理和机器学习的裂缝检测算法,必须人工提取HOG\citing{HOG}等特征,必须基于专家设计和人工辅助,无法实现智能化算法,使用场景不具备通用性。
虽然基于图像处理和机器学习的算法,模型尺寸小,算法开销小,但是检测精度不够,无法较好地适应实际的检测需求。

\subsection{基于新兴深度学习的裂缝检测系统}

近年来,基于深度学习的裂缝检测算法在智能化和检测精度上,具有较大优势,
业界在这方面的研究和探索颇多\citing{crackdetect_yolov2, uav_crackdetect, smartphone_crackdetect, attention_crackdetect, micro_pc_crackdetect, duikang_crackdetect, corr_crackdetect, conv_tower_crackdetect}。

Vishal Mandal等人\citing{crackdetect_yolov2}提出了基于YOLOv2算法的裂缝自动检测和识别系统。
使用7,240张图像训练系统,并在1,813张道路图像上进行测试,识别准确性等结果验证了深度学习算法在裂缝检测工程应用上的可行性。

2017年袁亚超等人\citing{uav_crackdetect}将基于Faster R-CNN的目标检测算法部署在无人机上进行裂缝检测,设计并实现了一套可用于实际工程的路面裂缝检测和定位系统。
该系统在一定程度上实现了对道路裂缝的快速、准确的采集、识别、定位和展示,为产业化应用提供了相关技术路线参考。

Hiroya Maeda等人\citing{smartphone_crackdetect}主要贡献在于以下两点:
一是创建了一个大规模的使用智能手机拍摄的道路损坏数据集,并使用卷积神经网络训练了一个损坏检测模型,将损坏类型分类为八类,展示了准确率。
二是对比了MobileNet和Inception V2结合SSD构建的轻量卷积神经网络,分别在PC端GPU和智能手机算力条件下的推理速度,结果显示GPU上推理速度最高达到30ms/帧,
在智能手机上推理速度达到1500ms/帧,1.5s的推理速度展现了其在成本低、算力资源受限设备上实际应用的可行性。

谢安东\citing{attention_crackdetect}使用改进的混合注意力机制、特征融合方法,融合YOLOV5目标检测算法,建立了裂缝检测
模型。实验结果显示,相比YOLOV5,检测性能得到进一步提升。

张旭\citing{micro_pc_crackdetect}将YOLOV3和Tiny-YOLOV3算法分别部署在云端服务器和OpenVINO微型终端,并检测两种算法和部署方案的检测精度和检测效率。

胡俊芳\citing{duikang_crackdetect}提出一种基于对抗和深度引导网络的路面裂缝检测方法。
实验结果显示,该算法准确率、召回率、精确率高于其他算法,并基于此算法搭建了一个路面裂缝检测系统。

基于新兴深度学习算法的裂缝检测系统,在检测精度、抗干扰性、鲁棒性等性能上取得很大提升,但也存在模型尺寸大、参数多、计算量大、推理慢等问题。


\section{本文主要工作}
本文针对道路裂缝检测实际场景精度和速度的双重需求,基于U-Net卷积神经网络及其端侧部署,设计并实现了一套基于边缘智能的道路裂缝检测系统。
具体的工作内容如下:

(1)针对裂缝检测,设计和实现了基于U-Net的卷积神经网络模型,并完成了模型参数训练。基于实验验证了该模型实现裂缝图像分割和检测的有效性。

(2)在原有裂缝图像数据集基础上,应用仿射变换和色彩变换等数据增广技术,扩大了数据集数量和增加数据集的多样性。通过实验对比分析,验证数据增广技术将模型得分提升到了0.93左右,大幅增强了模型的泛化能力和鲁棒性。

(3)针对训练完成的U-Net模型,完成了从float32位到int8位的模型参数量化,模型大小减小为原来的1/4。实验结果显示,模型量化前后推理结果余弦相似度超过0.99,精度损失极小。

(4)基于量化后的U-Net模型,在低成本、低功耗的边缘芯片海思Hi3516DV300上完成了模型的部署、推理和裂缝图像的实时检测。
系统测试显示,裂缝检测功能正常,且推理速度达到了0.5秒每帧,验证了裂缝检测算法在边缘设备部署的可行性,为裂缝检测的产业化应用提供了一般技术路线参考和启发。

本文创新点在于:

(1)在边缘设备上部署裂缝检测算法,可以有效避免云端部署带来的通信时延、实时检测不稳定等潜在问题,提高了系统检测的速度和稳定性。

(2)在端侧执行模型推理任务,无需将采集的图像上传,可以有效减轻中心服务器的负担和保护重要地区的数据隐私。

(3)设计并实现了一套低功耗、低成本的边缘智能裂缝检测系统,且取得精度和速度的较好平衡,有利于降低裂缝检测系统的开发和使用成本,促进其进一步的产业化应用。


\section{本文组织安排}
第一章介绍了研究背景、国内外裂缝检测算法及系统的发展现状,阐述了本文的主要工作及创新点。

第二章对行文所涉及的相关基础理论和技术原理进行了详细讲解。

第三章设计并实现了基于U-Net的边缘智能裂缝检测算法。包括阐述了U-Net的整体结构和关键特性,
完成了基于仿射变换和色彩变换的数据增广、模型参数训练以及在测试集上的精度测试、对比和结果分析。

第四章基于前文训练完成的算法模型,完成了在边缘设备中的模型量化、部署和实时推理。设计实现了边缘智能裂缝检测系统,并通过系统性能测试验证了其工程应用的可行性。

第五章是总结与展望,对全文内容进行了总结,阐述了本文实现的裂缝检测系统的应用价值,并提出了自身不足之处和未来展望,

\section{本章小结}
本章介绍了裂缝检测的研究背景,阐述了裂缝检测算法及系统的国内外研究现状,并给出了裂缝检测系统设计与实现的主要工作、创新点与章节组织安排。
