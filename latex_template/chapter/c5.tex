\chapter{总结与展望}

\section{总结}
针对当前道路裂缝检测系统精度和速度提升的双重需求,本文设计一套基于U-Net的边缘智能裂缝检测系统。
该系统在保持一定算法精度的前提下,将模型量化、部署在边缘设备平台上,
从而有效避免了云端部署带来的通信时延和不稳定性,有效提高了系统的检测速度和稳定性。

经过算法精度实验,验证了U-Net卷积分割网在路面裂缝检测上的可行性和检测效果。
系统性能测试显示,裂缝检测算法在端侧部署,取得了较好的精度和速度的平衡,为产业化应用提供了参考价值。

该系统不仅降低了系统的开发和使用成本,而且具有减轻中心服务器负担和保护重要地区数据隐私的优势。

\section{展望}
道路裂缝检测对于路面维护、避免潜在行车风险等具有重要意义,
本文提出了一套基于U-Net卷积网的边缘智能道路裂缝检测系统。

该系统的关键工作是把参数量、计算量大的深度学习裂缝检测算法部署在低成本、低功耗、资源受限的边缘设备上,这为裂缝检测推进产业化应用提供了参考价值。
但是,仍然存在以下不足和改进方向:

(1)本文使用的数据集数量少,而且背景较为简单。实际道路裂缝检测采集的图像中可能存在复杂的街景。因此采集并制作背景更复杂、干扰因素更多、数量更庞大的裂缝图像数据集,对于提升模型泛化能力和鲁棒性十分重要。

(2)本文的边缘系统裂缝检测推理速度约为0.5秒/帧,这要求道路检测车不能行驶速度过快。因此,采用成本更高、算力更强的终端设备进行算法部署,
或者通过算子优化、计算图优化、编译优化等方法,进一步提升端侧模型推理速度,对于提高检测车工作效率十分重要。

(3)本文采用的裂缝检测是分割网,对裂缝和背景进行分类和像素级别的分割,虽然具有精度更高的裂缝识别优势,但其模型参数量和计算量相比业界常用的检测网更大,推理效率有所下降。
因此,考虑将检测网和分割网结合使用,可能是一个发展方向。当检测到裂缝时,再将其框选、裁剪送入分割网进行像素水平的分割,既保证了效率又保留了精度。