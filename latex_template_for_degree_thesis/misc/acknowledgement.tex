\thesisacknowledgement
行文至此,大学生涯即将落下帷幕。从小镇青年到易海湖畔,在这里,我度过了人生最重要的四年时光。
这四年,是我成长最多的四年。
有坚定的时候,也有迷茫的时候;
有充实的时候,也有虚无的时候。
所幸,山穷水复疑无路,柳暗花明又一村。
在这四年里,我完成了从否定到接纳、从小我到大我的重要转变。

首先,要感谢党和国家,感谢辛勤、善良的中国人民。
就在我行文之时,世界上还有很多地方饱受战争、贫困、饥饿的痛苦,我很庆幸自己生在华夏,在一个和平的国家,无所顾忌地享受安宁的大学学习和生活。
希望毕业后的自己,继续学习、进步,用自己的专业所学,为中华民族复兴、发展智能科技、探索世界真理、推动人类进步做出一份微薄的贡献!

其次,要特别感谢我的毕业设计指导老师苏亮亮老师。
在整个毕设指导过程中,苏老师多次给我们开组会,针对我们的开题报告、中期检查、论文写作等方面给予了详细的指导。
还记得一次周末的早上苏老师给我们详细地讲解论文写作格式和表达规范,让我第一次深刻地感受到了科学研究、学术论文写作的严谨性和规范性。
很高兴能够成为苏老师的学生,在他的指导下,完成了我的本科毕业论文。

另外,还要特别感谢
邵慧老师、肖晓老师、徐恒老师、孙光灵老师、邓静老师、谢娟老师、吴瀛老师、汪淼老师等(排名不分先后)学科竞赛指导老师。
各位老师积极组织我们参赛、备赛,为我们提供比赛指导,在此十分感谢他们的辛勤付出。
以赛促学是最好的一种能力增长的方式,在此还要特别感谢学校、学院,一直为我们提供资源平台并鼓励我们积极参赛。

其次,还要感谢丁宇、范奚鸣、胡傲、田昊华、金许涵、钱敏、王步历、张楠同学(排名不分先后)以及我的室友们。
他们都是德智体美劳全面发展的新时代的好青年。
有他们在身边,我成为了更好的自己。
感谢他们!一日为友,终生为友!

最后,我要特别感谢我的父母家人。我的家人,支持我走过了十几年的学生生涯。
学习的日子里,有挫折、有失败;有成功、有喜悦。
他们都希望我不要压力太大,始终相信我、鼓励我,让我拥有了一个宽松、自由、温馨的成长环境。
希望我的父母、我的家人们都能身体健康、幸福快乐!

人生的意义在于享受过程、一路前行,在于为人民服务,为家人、为朋友、为社会做出自己的一份贡献。
祝福大家身体健康、学业有成、工作顺利,成为一个幸福的人!

