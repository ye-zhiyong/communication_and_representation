	
\begin{chineseabstract}
道路裂缝检测是公路养护、保证行车安全的重要工作内容。
基于深度学习技术的检测方法相比传统方法,具有检测精度更高的优势。
截止目前,道路裂缝检测系统依然存在以下问题:
1)算法上业界多追求精度提升,导致模型越做越大,降低了检测速度,影响了工作效率。
2)系统上模型多部署在云端,将采集的图像上传至云端进行推理,存在通信时延、效率低、不稳定、中心服务器过载等问题。


本文针对道路裂缝检测工程应用中精度和速度的双重需求,基于U-Net卷积网及其端侧部署,设计并实现了一套基于边缘智能的道路裂缝检测系统。
主要研究内容如下:
1)设计实现了基于U-Net卷积网的路面裂缝检测算法,并利用数据增广技术完成了模型参数训练,有效提升了检测精度。实验显示,数据增广后模型分割的F1得分提升到0.93左右,验证了算法用于裂缝检测的有效性。
2)基于训练完成的U-Net模型,实现了从float32位到int8位的模型参数量化,模型大小减小为原来的1/4,推理速度得到提升。实验结果显示,基于若干样本模型量化前后的推理结果向量,计算获得的余弦相似度平均超过0.99,精度损失极小。
3)基于量化后的U-Net模型,在低成本、低功耗的边缘芯片海思Hi3516DV300上完成了模型的部署、推理和裂缝图像的实时检测。
系统性能测试显示,裂缝检测功能正常,且推理速度接近0.5秒/帧,取得精度和速度的较好平衡,验证了裂缝检测算法在边缘系统上部署的可行性。


该边缘智能裂缝检测系统,不仅有效避免了通信时延、提高了稳定性;
而且减轻了中心服务器的负担、降低了系统成本,为裂缝检测的产业化应用提供了参考。

\chinesekeyword{边缘智能;裂缝检测;深度学习;卷积神经网络;U-Net}
\end{chineseabstract}

